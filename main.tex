%%%%%%%%%%%%%%%%%
% This is an sample CV template created using altacv.cls
% (v1.1.5, 1 December 2018) written by LianTze Lim (liantze@gmail.com). Now compiles with pdfLaTeX, XeLaTeX and LuaLaTeX.
%
%% It may be distributed and/or modified under the
%% conditions of the LaTeX Project Public License, either version 1.3
%% of this license or (at your option) any later version.
%% The latest version of this license is in
%%    http://www.latex-project.org/lppl.txt
%% and version 1.3 or later is part of all distributions of LaTeX
%% version 2003/12/01 or later.
%%%%%%%%%%%%%%%%

%% If you need to pass whatever options to xcolor
\PassOptionsToPackage{dvipsnames}{xcolor}

%% If you are using \orcid or academicons
%% icons, make sure you have the academicons
%% option here, and compile with XeLaTeX
%% or LuaLaTeX.
% \documentclass[10pt,a4paper,academicons]{altacv}

%% Use the "normalphoto" option if you want a normal photo instead of cropped to a circle
% \documentclass[10pt,a4paper,normalphoto]{altacv}

\documentclass[10pt,a4paper,ragged2e]{altacv}

%% AltaCV uses the fontawesome and academicon fonts
%% and packages.
%% See texdoc.net/pkg/fontawecome and http://texdoc.net/pkg/academicons for full list of symbols. You MUST compile with XeLaTeX or LuaLaTeX if you want to use academicons.

% Change the page layout if you need to
\geometry{left=1cm,right=9cm,marginparwidth=6.8cm,marginparsep=1.2cm,top=1.25cm,bottom=1.25cm}

% Change the font if you want to, depending on whether
% you're using pdflatex or xelatex/lualatex
\ifxetexorluatex
  % If using xelatex or lualatex:
  \setmainfont{Carlito}
\else
  % If using pdflatex:
  \usepackage[utf8]{inputenc}
  \usepackage[T1]{fontenc}
  \usepackage[default]{lato}
\fi

% Change the colours if you want to
\definecolor{Red}{HTML}{da3041}
\definecolor{SlateGrey}{HTML}{2E2E2E}
\definecolor{LightGrey}{HTML}{666666}
\colorlet{heading}{Sepia}
\colorlet{accent}{Red}
\colorlet{emphasis}{SlateGrey}
\colorlet{body}{LightGrey}

% Change the bullets for itemize and rating marker
% for \cvskill if you want to
\renewcommand{\itemmarker}{{\small\textbullet}}
\renewcommand{\ratingmarker}{\faCircle}

%% sample.bib contains your publications
\addbibresource{sample.bib}

\begin{document}
\name{Ryan MacGillivray}
\tagline{Lead Software Engineer and occasional beer brewer}
\photo{2.8cm}{ryan}
\personalinfo{%
  % Not all of these are required!
  % You can add your own with \printinfo{symbol}{detail}
  \email{rmacg89@gmail.com}
  \phone{07444149041}
  \mailaddress{22A, Denmark Hill, London}
  \location{London, UK}
  \twitter{@RyanMacG}
  \linkedin{linkedin.com/in/ryanmacgillivray}
  \github{github.com/ryanmacg}
  %% You MUST add the academicons option to \documentclass, then compile with LuaLaTeX or XeLaTeX, if you want to use \orcid or other academicons commands.
  % \orcid{orcid.org/0000-0000-0000-0000}
}

%% Make the header extend all the way to the right, if you want.
\begin{fullwidth}
\makecvheader
\end{fullwidth}

%% Depending on your tastes, you may want to make fonts of itemize environments slightly smaller
% \AtBeginEnvironment{itemize}{\small}

%% Provide the file name containing the sidebar contents as an optional parameter to \cvsection.
%% You can always just use \marginpar{...} if you do
%% not need to align the top of the contents to any
%% \cvsection title in the "main" bar.
\cvsection[page1sidebar]{Experience}

\cvevent{Lead Software Engineer}{Bit Zesty}{Dec 2018 -- Ongoing}{London}
\begin{itemize}
\item Meeting with customers to help roadmap their needs and strengthen relationships
\item Coordinating other Engineers on projects to provide support and ensure timely deliveries for customers
\item Developing features for multiple existing customers
\end{itemize}

\divider

\cvevent{Software Engineer}{Zen Educate}{Jul 2018 -- Dec 2018}{London}
\begin{itemize}
\item Introducing code quality tools to drive up quality
\item Leading on introduction of greater test coverage and Test Driven Development
\item Building engineering team's experience in running Agile Ceremonies such as retrospectives
\item Working with internal stakeholders to determine their needs and developing features to meet these
\end{itemize}

\divider

\cvevent{Software Engineer}{Made Tech}{Aug 2016 -- Jul 2018}{London}
\begin{itemize}
\item Led the initial Academy program and reworked curriculum for future years
\item Worked on teams delivering both greenfield projects and supporting existing codebases in a variety of languages/frameworks
\item Mentoring of junior developer (later hired at Made Tech)
\end{itemize}

\divider

\cvevent{Software Engineer}{Currency Cloud}{Nov 2015 -- Aug 2016}{London}
\begin{itemize}
\item Working on FinTech platform comprising of JRuby, Sinatra and Rails applications
\item Migration of Sinatra application off of HornetQ to RabbitMQ
\item Migration of Sinatra application off of Torquebox to modern application server
\end{itemize}

\clearpage

\cvsection[page2sidebar]{Experience Contd.}
\cvevent{Software Engineer}{tictoc}{Jul 2013 -- Nov 2015}{Glasgow}
\begin{itemize}
\item Lead developer on several larger client sites including SCO, B-Eat and GAP Group
\item Lead developer on an unreleased product:
\begin{itemize}
\item Converted several costly actions to background jobs using ActiveJob, Sidekiq and the Services pattern
\item Established standard JSON format for several disparate social APIs
\item Developed initial prototype and client facing implementation of project
\item Developed admin backend for product
\end{itemize}
\item Helped to develop and maintain proprietary Rails 4 based CMS.
\end{itemize}

\divider

\cvevent{IT Support Engineer/Web Developer}{JTC Furniture Group}{Jun 2011 -- Jul 2013}{Dundee}

\cvsection{A Day of My Life}

% Adapted from @Jake's answer from http://tex.stackexchange.com/a/82729/226
% \wheelchart{outer radius}{inner radius}{
% comma-separated list of value/text width/color/detail}
\wheelchart{1.5cm}{0.5cm}{%
  6/8em/accent!30/{Sleeping I guess},
  3/8em/accent!40/Training to play Mario Kart competitively,
  8/8em/accent!60/Day job,
  2/10em/accent/Cycling or watching wrestling,
  5/6em/accent!20/Spending time with my wife or friends
}

%% If the NEXT page doesn't start with a \cvsection but you'd
%% still like to add a sidebar, then use this command on THIS
%% page to add it. The optional argument lets you pull up the
%% sidebar a bit so that it looks aligned with the top of the
%% main column.
% \addnextpagesidebar[-1ex]{page3sidebar}


\end{document}
